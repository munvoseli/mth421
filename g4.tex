\documentclass[12pt]{article}

%\AtEndDocument{\cleardoublepage}
%\AtEndDocument{% based in \cleardoublepage
%  \clearpage
%  \ifodd\value{page}\else
%    \thispagestyle{empty}%
%    \hbox{}\newpage
%  \fi
%}

\usepackage[margin=1.0in]{geometry} 
\usepackage{amsmath,amsthm,amssymb,amsfonts}
\usepackage{fancyvrb}
\usepackage{graphics}

\usepackage{enumitem}

\theoremstyle{definition}
\newtheorem*{prob}{Problem}
\newtheorem*{soln}{Solution}

\DeclareMathOperator{\lcm}{lcm}
\newcommand{\hr}{\vspace*{\parskip}\hrule}

\newcommand{\CC}{{\mathbb{C}}}
\newcommand{\ZZ}{{\mathbb{Z}}}
\newcommand{\RR}{{\mathbb{R}}}
\newcommand{\NN}{{\mathbb{N}}}
\newcommand{\QQ}{{\mathbb{Q}}}
\newcommand{\PP}{{\mathbb{P}}}


\renewcommand{\mod}{\operatorname{mod}}

\begin{document}

% T I T L E
%
\title{421 HW 4 Group}
\author{\textbf{change this to your names}}

\date{}

\maketitle

\textbf{You may use Subring Test Theorem 3.6 for your justifications in Section 3.1.}


%\begin{prob}[3.1.27] Let $S$ be the set of rational numbers that can be written
%with an odd denominator. Prove that $S$ is a subring of $\QQ$, but is not a field.
%\end{prob}
%
%\begin{soln}
%
%\end{soln}


%\begin{prob}[3.1.32] 
%Let $R$ be a ring and let $Z(R) = \{ a \in R \mid ar = ra \text{ for every } r \in R \}$.
%In other words, $Z(R)$ consists of all elements of $R$ that commute with every other
%element of $R$. Prove that $Z(R)$ is a subring of $R$.
%\end{prob}
%
%\begin{soln}
%
%\end{soln}


\begin{prob}[3.2.13]
Let $S$ and $T$ be subrings of a ring $R$. In (a) and (b), if  
the answer is ``yes,'' prove it. If  the answer is ``no,'' give a counterexample.

(a) Is $S \cap T$ a subring of $R$?

(b) Is $S \cup T$ a subring of $R$?
\end{prob}

\begin{soln}

(a)

Let $a$ be an element of $S\cap T$, and let $b$ be an element of $S\cap T$.

\renewcommand\land{\text{ and }}
Since $a\in S\cap T$ and $b\in S\cap T$, $a\in S\land a\in T$, and $b\in S\land b\in T$.

Since $a\in S\land b\in S$, $ab\in S\land a+b\in S$ since $S$ is a ring.
Similarly, $ab\in T\land a+b\in T$.
Since $ab\in S\land ab\in T$, $ab\in S\cap T$, so $S\cap T$ is closed under multiplication.
Similarly, $a+b\in S\land a+b\in T$, so $a+b\in S\cap T$, so $S\cap T$ is closed under addition.

Since $S\neq\emptyset$ (because $S$ is a ring), there exists $s\in S$.
Since $s\in S$, $s$ has an additive inverse $-s$.
$s+(-s)=0_R$ (because addition in $S$ is the same as addition in $R$),
so, by closure of addition, $0_R\in S$.
Similarly, $0_R\in T$.
Therefore, $0_R\in S\cap T$, so $S\cap T$ is nonempty.

By the Subring Test Theorem, $S\cap T$ is a subring of $R$.

(b)

$\{0, 2, 4\}$ and $\{0, 3\}$ are both subrings of $\ZZ_6$.
However, $\{0, 2, 3, 4\}$ is not closed under addition
($2+3=5$, $2+3\not\in\{0,2,3,4\}$),
so not necessarily a subring.

\end{soln}

\hr


%\begin{prob}[3.2.22] (a) If $ab$ is a zero divisor in a ring $R$, prove that
%$a$ or $b$ is a zero divisor.
%
%(b) If $a$ or $b$ is a zero divisor in a commutative ring $R$ and $ab \neq 0_R$,
%prove that $ab$ is a zero divisor.
%\end{prob}
%
%\begin{soln}
%
%
%\end{soln}


\begin{prob}[3.2.25] Let $S$ be a subring of a ring $R$ with identity.

(a) If $S$ has an identity, show by example that $1_S$ may not be the same as $1_R$.

(b) If both $R$ and $S$ are integral domains, prove that $1_S=1_R$.
\end{prob}

\begin{soln} (a)

$\ZZ_6$ has an identity $1$, but its subring consisting of elements
$\{0, 2, 4\}$ has $4$ as a multiplicative identity.

(b)

Since $S$ is a subring of $R$, $1_S\in R$.


Let $S$ and $R$ be integral domains. Then there exist $1_S \in S$ and $1_R \in R$.
For some $a\in S^*$ (which is nonempty, since by definition of integral domain, there exists some element $1_S\neq0_S$ and $1_S\in S$), $1_S(a)=(a)1_S=a$.
Additionally, since $S^*$ is contained in $R$, $a\in R$, so $1_R\cdot a=a\cdot 1_R=a$.

\begin{align*}
1_S(a)&=a\\
1_R(a)&=a\\
1_S(a)&=1_R(a)\\
1_S\cdot a+(-1_R)\cdot a&= 0\\
(1_S+(-1_R))a&=0\\
1_S+(-1_R)&=0\text{ (because integral domain)}\\
1_S&=1_R
\end{align*}

\end{soln}


\hr

%\begin{prob}[3.2.26]
%Let $S$ be a subring of  a ring $R$. Prove that $0_S = 0_R$. 
%[Hint: For fixed $a,b \in R$, how many solutions $x \in R$ are there for $a + x = b$?
%Now consider $a=b=0_S$.]
%\end{prob}
%
%\begin{soln}
%
%\end{soln}


\begin{prob}[3.2.31]
A \textbf{Boolean ring} is a ring $R$ with identity in which 
$x^2 = x$ for every $x\in R$. For examples, see Exercises 19 and 44 in Section 3.1. 
If $R$ is a Boolean ring, prove that      

(a)   $a + a = 0_R$ for every $a\in R$, which means that $a = -a$. 
[Hint: Expand $(a + a)^2$.]     

(b) $R$ is commutative. [Hint: Expand $(a + b)^2$.]
\end{prob}

\begin{soln}

(a)

\begin{align*}
(a+a)^2&=(a+a)^2\\
a^2+a^2+a^2+a^2&=a+a\\
a+a+a+a&=a+a\\
a+a&=0\\
a&=-a
\end{align*}

(b)

\begin{align*}
(a+b)^2&=(a+b)^2\\
a+b&= (a+b)(a+b)\\
a+b&=a^2+ab+ba+b^2\\
a+b&=a+ab+ba+b\\
0&=ab+ba\\
-ba&=ab\\
ba&=ab
\end{align*}

\end{soln}

\hr

\begin{prob}[3.2.DK1]
Let $R$ be a ring with identity. Prove that if $1_R=0_R$, then
$R = \{ 0_R \}$. That is, $R$ is the zero ring.
\end{prob}

\begin{soln}

Let $a$ be any element of $R$.

\begin{align*}
0a&=0\text{ by Theorem 3.5}\\
1a&=a\\
0&=1\\
0a&=1a\\
0&=a
\end{align*}

Any element $a$ is equal to $0$,
so every element of $R$ is equal to $0$.
Since all elements are $0$, $R=\{0_R\}$.

\end{soln}

\end{document}
