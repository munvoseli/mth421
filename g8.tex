\documentclass[12pt]{article}

%\AtEndDocument{\cleardoublepage}
%\AtEndDocument{% based in \cleardoublepage
%  \clearpage
%  \ifodd\value{page}\else
%    \thispagestyle{empty}%
%    \hbox{}\newpage
%  \fi
%}

\usepackage[margin=1.0in]{geometry} 
\usepackage{amsmath,amsthm,amssymb,amsfonts}
\usepackage{fancyvrb}
\usepackage{graphics}

\usepackage{enumitem}

\theoremstyle{definition}
\newtheorem*{prob}{Problem}
\newtheorem*{soln}{Solution}

\DeclareMathOperator{\lcm}{lcm}

\newcommand{\CC}{{\mathbb{C}}}
\newcommand{\ZZ}{{\mathbb{Z}}}
\newcommand{\RR}{{\mathbb{R}}}
\newcommand{\NN}{{\mathbb{N}}}
\newcommand{\QQ}{{\mathbb{Q}}}
\newcommand{\PP}{{\mathbb{P}}}


\renewcommand{\mod}{\operatorname{mod}}

\begin{document}

% T I T L E
%
\title{421 HW 8 Group}
\author{\textbf{change this to your names}}

\date{}

\maketitle

\textbf{NOTE}: Unless stated otherwise, $G$ is a (multiplicative) 
group with identity element $e$.

\begin{prob}[7.4.26]
If  $G = \langle a\rangle$ is a cyclic group and $f:G \to H$ is a surjective 
homomorphism of  groups, show that $f(a)$ is a generator of $H$, 
that is, $H$ is the cyclic group $\langle f(a)\rangle$. [Hint: Exercise 7.4.15.]
\end{prob}

\begin{soln}

By 7.4.15, $f(a)^n=f(a^n)$.

Since $f$ is surjective, $\forall h\in H\quad \exists g\in G\quad f(g)=h$.

Since $g\in G$, $g=a^n$ for some $n$, and thus
\begin{align*}
f(g)&=h\\
f(a^n)&=h\\
f(a)^n&=h
\end{align*}

Every member of $H$ is some power of $f(a)$.

\end{soln}

%\begin{prob}[7.4.29]
%If  $f:G \to H$ is an injective homomorphism of groups and $a\in G$, 
%prove that  $|f(a)| = |a|$. [Remember there are finite and infinite cases.]
%\end{prob}
%
%\begin{soln}
%
%\end{soln}


\begin{prob}[7.4.30]
Let $f:G \to H$ be a homomorphism of groups and let $J$ be a subgroup of $H$. 
Prove that the set $L=\{a\in G \mid f(a)\in J\}$ is a subgroup of $G$.
[Note: The set $L$ is the \emph{inverse image} of $J$. You may have seen the
notation $L=f^{-1}(J)$. Exercise 7.4.33 is the special case $J = \{ e_H \}$.
That is, $K_f = f^{-1}(\{ e_H \})$, the \emph{kernel} of $f$. 
I changed some of the letters in this
exercise so as not to conflict with Exercise 7.4.33.]
\end{prob}

\begin{soln}

To prove: $L$ is nonempty

$J$ has $e_H$ since it's a subgroup of $H$.

$f(e_G)$ must be $e_H$ as a result of $f$ being a homomorphism.

Since $f(e_G)\in L$, $L$ is nonempty.

To prove: $L$ is closed under multiplication

Let $a, b\in L$.  Since $f$ is a homomorphism, $f(a)f(b)=f(ab)$.
By the definition of $L$, $f(a), f(b)\in J$.
Since $J$ is a group, $f(ab)\in J$, and $ab\in L$.

To prove: $L$ is closed under inverses

Let $a\in L$.
By the definition of $L$, $f(a)\in J$.
Since $J$ is a group, $f(a)^{-1}\in J$.
As a result of $f$ being a homomorphism, $f(a^{-1})\in J$.
By the definition of $L$, $a^{-1}\in L$.

\end{soln}

\begin{prob}[7.4.34]
The function $f:\ZZ \to \ZZ_5$ given by $f(x) = [x]$ is a homomorphism by Example 13. 
Find $K_f$ 
and justify your finding. (Notation as in Exercise 7.4.33.)
\end{prob}

\begin{soln}

Exercise 33 says $K_f=\{a\in G\ |\ f(a)=e_H\}$

In this case, $K_f=\{a\in \ZZ\ |\ [a]_5=[0]_5\}$

$K_f$ is the set of all integer multiples of 5.

If an integer $k$ is a multiple of 5, then $5\mid k-0$, and $[k]_5=[0]_5$.
If an integer $k$ is not a multiple of 5, then,
for some $b\in\{1,2,3,4\}$, $5\mid k-b$,
$[k]_5=[b]_5$, where $[b]_5\neq [0]_5$.
By transitivity of equivalence, $[k]_5\neq [0]_5$.
Therefore, $k\in K_f$ if and only if $5\mid k$.

\end{soln}




\begin{prob}[7.4.53]
Let $f:G \to H$ be an isomorphism of  groups. Let $g:H \to G$ 
be the inverse function of  $f$ as defined in Appendix B. 
Prove that $g$ is also an isomorphism of  groups. 
(You may assume that the inverse of a bijection is a bijection, since
you hopefully saw this in MTH 331.)
[Hint: To show that $g(ab) = g(a)g(b)$, 
consider the images of  the left-and right-hand sides under $f$ and use the facts 
that $f$ is a homomorphism and $f \circ g$ is the identity map.]
\end{prob}

\begin{soln}

\end{soln}

\begin{prob}[7.4.DK2] Let $\cong$ represent isomorphism between groups.
Prove that $\cong$ is an equivalence relation.
[Hint: See 7.4 Example 8, Individual Exercise 7.4.DK1, and Exercise 7.4.53]
\end{prob}

\begin{soln}

% reflexive: identity function
% symmetric:
% transitive:

\end{soln}



\end{document}
