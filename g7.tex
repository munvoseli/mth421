\documentclass[12pt]{article}

%\AtEndDocument{\cleardoublepage}
%\AtEndDocument{% based in \cleardoublepage
%  \clearpage
%  \ifodd\value{page}\else
%    \thispagestyle{empty}%
%    \hbox{}\newpage
%  \fi
%}

\usepackage[margin=1.0in]{geometry} 
\usepackage{amsmath,amsthm,amssymb,amsfonts}
\usepackage{fancyvrb}
\usepackage{graphics}

\usepackage{enumitem}

\theoremstyle{definition}
\newtheorem*{prob}{Problem}
\newtheorem*{soln}{Solution}

\DeclareMathOperator{\lcm}{lcm}

\newcommand{\CC}{{\mathbb{C}}}
\newcommand{\ZZ}{{\mathbb{Z}}}
\newcommand{\RR}{{\mathbb{R}}}
\newcommand{\NN}{{\mathbb{N}}}
\newcommand{\QQ}{{\mathbb{Q}}}
\newcommand{\PP}{{\mathbb{P}}}


\renewcommand{\mod}{\operatorname{mod}}

\begin{document}

% T I T L E
%
\title{421 HW 7 Group}
\author{\textbf{change this to your names}}

\date{}

\maketitle

\textbf{NOTE}: Unless stated otherwise, $G$ is a (multiplicative) 
group with identity element $e$.



\begin{prob}[7.2.8]
Give an example of  a group that contains nonidentity elements of  
finite order and of  infinite order. (That is, just one group
has both types of nonidentity elements.) Give an example of each type of element
in your chosen group.
\end{prob}

\begin{soln}\quad

$M_2(\RR)$

Of finite order:

\begin{equation*}
\left[
\begin{tabular}{c c}
1 & 0\\
0 & -1
\end{tabular}
\right]
\end{equation*}

Of infinite order:

\begin{equation*}
\left[
\begin{tabular}{c c}
2 & 0\\
0 & 2
\end{tabular}
\right]
\end{equation*}

\end{soln}

\begin{prob}[7.2.10]
Find the order of  every element in each group:\\
(a) $\ZZ_4$ \quad    (b) $\ZZ_4 \times \ZZ_2$ \quad    (c) $S_3$  
\quad    (d) $D_4$  \quad   (e) $\ZZ$
\end{prob}

\begin{soln}\quad

(a) $\ZZ_4$

\begin{equation*}
\begin{tabular}{c c}
0 & 1\\
1 & 4\\
2 & 2\\
3 & 4
\end{tabular}
\end{equation*}

(b) $\ZZ_4\times \ZZ_2$

\begin{equation*}
\begin{tabular}{c c}
(0, 0) & 1\\
(1, 0) & 4\\
(2, 0) & 2\\
(3, 0) & 4\\
(0, 1) & 2\\
(1, 1) & 4\\
(2, 1) & 2\\
(3, 1) & 4
\end{tabular}
\end{equation*}

(c) $S_3$

\begin{equation*}
\begin{tabular}{c c}
(0, 1, 2) & 1\\
(0, 2, 1) & 2\\
(1, 0, 2) & 2\\
(2, 1, 0) & 2\\
(1, 2, 0) & 3\\
(2, 0, 1) & 3
\end{tabular}
\end{equation*}

(d) $D_4$

\begin{equation*}
\begin{tabular}{c c}
$r_0$ & 1\\
$r_1$ & 4\\
$r_2$ & 2\\
$r_3$ & 4\\
$h$ & 2\\
$v$ & 2\\
$d$ & 2\\
$$ & 2 % TODO
\end{tabular}
\end{equation*}

(e) $\ZZ$

0 has an order of 1.
All other elements of $\ZZ$ have infinite order.

\end{soln}

%\begin{prob}[7.2.13]
%If  $G$ is a finite group of  order $n$ and $a\in G$, prove that 
%$|a| \leq n$. 
%[Hint: Consider the $n + 1$ elements $e = a^0, a, a^2, a^3, \dots , a^n$. 
%Are they all distinct?] Thus every element in a finite group has finite order. 
%(Note: The converse, however, is false; see Exercise 25 in Section 8.3 for an infinite group 
%in which every element has finite order. You don't need to do anything about this note.)
%\end{prob}
%
%\begin{soln}
%
%\end{soln}

\begin{prob}[7.2.19]
If $a, b\in G$, prove that $|bab^{-1}| = |a|$. [Keep in mind there
are finite and infinite cases.]
\end{prob}

\begin{soln}\quad

Lemma: $(bab^{-1})^n=ba^nb^{-1}$ for all $n\in\ZZ^+$

Base case: $(bab^{-1})^1=ba^1b^{-1}$

Inductive case:
\begin{gather*}
(bab^{-1})^n=ba^nb^{-1}\\
(bab^{-1})^nbab^{-1}=ba^nb^{-1}bab^{-1}\\
(bab^{-1})^nbab^{-1}=ba^nab^{-1}\\
(bab^{-1})^{n+1}=ba^{n+1}b^{-1}\\
\end{gather*}

The infinite case:

Suppose $|a|=\infty$.
This means that, for all $n\in\ZZ^+$, $a^n\neq e$.
%TODO turn into english

\begin{gather*}
|a|=\infty\\
\Updownarrow\\
\forall n\in\ZZ^+\quad a^n\neq e\\
\Updownarrow\\
\forall n\in\ZZ^+\quad ba^nb^{-1}\neq bb^{-1}\\
\Updownarrow\\
\forall n\in\ZZ^+\quad ba^nb^{-1}\neq e\\
\Updownarrow (a^3=ab^{-1}bab^{-1}ba)\\
\forall n\in\ZZ^+\quad (bab^{-1})^n\neq e\\
\Updownarrow\\
|bab^{-1}|=\infty
\end{gather*}

The finite case:

\begin{gather*}
\left|a\right|=n\\
a^n=e\\
ba^nb^{-1}=e\\
(bab^{-1})^n=e\\
\left|bab^{-1}\right|\mid n\\
\left|bab^{-1}\right|\mid \left|a\right|\\
\end{gather*}

\begin{gather*}
\left|bab^{-1}\right|=m\\
(bab^{-1})^m=e\\
ba^mb^{-1}=e\\
a^m=e\\
\left|a\right| \mid m\\
\left|a\right| \mid \left|bab^{-1}\right|\\
\end{gather*}

\begin{gather*}
\left|a\right| \mid \left|bab^{-1}\right|\\
\left|bab^{-1}\right|\mid \left|a\right|\\
\left|a\right|\in \ZZ^+\\
\left|bab^{-1}\right| \in\ZZ^+\\
\left|bab^{-1}\right|=\left|a\right|
\end{gather*}

\end{soln}

\begin{prob}[7.2.32]
If  $|G|$ is even, prove that $G$ contains an element of  order $2$. 
[Hint: The identity element is its own inverse. See the hint for Exercise 27.]
\end{prob}

\begin{soln}\quad

Since $|G|$ is even,
let the order of $G$ be expressed as $2n$, for some $n\in\ZZ^+$.

For contradiction, suppose that no elements of $G$ are of order $2$.

(less formally: put all elements of $G$ into pairs of inverses.
there will be two elements left: the identity element,
and one element which must be its own inverse (of order 2))

Let $S_{2n-1}$ be the set of all nonidentity elements of $G$.

For all $m\in[1,n)\cap\ZZ$, let $a$ be some element of $S_{2m+1}$.

\end{soln}

\begin{prob}[7.2.36]
Suppose $a, b\in G$ with $|a| = 5$, $b \neq e$, and $aba^{-1} = b^2$. 
Find $|b|$.
\end{prob}

\begin{soln}

\begin{align*}
aba^{-1}&=b^2\\
b&=a^4b^2a\\
b&=a^4bba\\
&=a^4(a^4bba)(a^4bba)a\\
&=a^3b^4a^2\\
&=...\\
&=a^5b^{32}a^5\\
&=eb^{32}e\\
&=b^{32}\\
b&=b^{32}\\
b^{-1}b&=b^{-1}b^{32}\\
e&=b^{31}\\
|b|&\mid 31\\
&\text{since $b\neq e$, $|b|\neq 1$}\\
|b|&=31
\end{align*}

\end{soln}

\begin{prob}[7.3.15] \text{ }
\textbf{(a)}   Let $H$ and $K$ be subgroups of a group $G$. 
Prove that $H \cap K$ is a subgroup of  $G$.  

\textbf{(b)}   Let $\{H_i\}$ be any collection of  subgroups of  $G$. 
Prove that $\cap H_i$ is a subgroup of  $G$.

Recall: If $\{ H_i \mid i \in I \}$ is a set of subsets of $G$, with indexing set $I$,
then, by definition, 
\[
\cap H_i = \cap_{i \in I} H_i = \{ a \in G \mid \forall i \in I, a \in H_i \}.
\]
\end{prob}

\begin{soln}

\end{soln}

\begin{prob}[7.3.26] \text{ }
\textbf{(a)}   Let $H$ and $K$ be subgroups of  an abelian group $G$ and let 
$HK = \{ab \mid a\in H, b\in K\}$. Prove that $HK$ is a subgroup of  $G$.

\textbf{(b)} Show that part (a) may be false if  $G$ is not abelian.
\end{prob}

\begin{soln}

\end{soln}

\end{document}
