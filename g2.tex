\documentclass{article}
\nonstopmode
\usepackage[paperwidth=20cm,paperheight=575cm]{geometry}
\usepackage[framemethod=TikZ]{mdframed}
\usepackage{amsmath, amsfonts, amssymb}
\usepackage{mathtools} % abs

\geometry{margin=1cm}
\DeclarePairedDelimiter\abs{\lvert}{\rvert}
\setlength{\parindent}{0pt}
\setlength{\parskip}{\baselineskip}
\newcommand{\hr}{\vspace*{\parskip}\hrule}
\newcommand{\ZZ}{\mathbb{Z}}
\DeclareRobustCommand{\eel}[1]{\begin{align*}{#1}\end{align*}}
\DeclareRobustCommand{\eelc}[1]{\begin{gather*}{#1}\end{gather*}}
\newmdenv[roundcorner=2pt,align=center%
,linecolor=yellow!20%
,backgroundcolor=yellow!20%
,linewidth=4,outerlinewidth=2%
,outerlinecolor=blue!70!black%
]{stonk}


\AddToHook{shipout/before}{%
    \pdfpageheight=\pagetotal
    \advance\pdfpageheight by 2cm
}
\let\oldsection=\section
\def\section{\pagebreak\oldsection}
\AtEndDocument{\pagebreak}


\begin{document}


\title{Group Work 2}
\author{Benjamin Belisle%
\and Ben Lannom%
\and Bennett Meacham%
\and Emma Owens}
\maketitle

\section*{1.3.34}

Suppose $n\in\ZZ$ and $n>2$.

Suppose also that $N$ is the set $\ZZ\cap[2,n]$.

$n!$ is the product of all integers within $N$,
so $\forall z\in N\ z\mid n!$.

\begin{stonk}

To prove that $a\nmid 1\land a\mid b\Rightarrow a\nmid b-1$

\hr

Suppose that $a\mid b$ and $a\mid b-1$.

This means that there are integers $c$ and $d$
such that $ac=b$ and $ad=b-1$.

\eel{
ac&=b\\
ad&=b-1\\
a(c-d)&=1
}
This means that $a\mid 1$.

\hr

Thus,
\eel{
a\mid b\land a\mid b-1\Rightarrow a\mid 1\\
\lnot(a\mid b\land a\mid b-1)\lor a\mid 1\\
a\nmid b\lor a\nmid b-1\lor a\mid 1\\
a\mid 1\lor a\nmid b\lor a\nmid b-1\\
a\nmid 1\land a\mid b\Rightarrow a\nmid b-1
}

\end{stonk}

For every $z\in N$, $z>1$, so $z\nmid 1$.

$z\nmid 1$ and $z\mid n!$, so $z\nmid n!-1$.

Thus, $n!-1$ has no factors $e$ such that $1<e\leq n$.
\eel{
1<e<\leq n\Rightarrow e\nmid n!-1\\
e\mid n!-1\Rightarrow (e\leq 1\lor e>n)
}

Since $n>2$, and factorial is increasing, $n!>2$, and $n!-1>1$.

Since $n!-1>1$, $n!-1$ has a prime factor $p$.

Since $p\mid n!-1$, $p\leq 1\lor p>n$.

Since $p$ is prime, it cannot be that $p\leq 1$,
so $p>n$.

Since $p\mid n!-1$ and $n!-1>0$, $p\leq n!-1$.

Thus, $n<p\leq n!-1$.

There exists a prime $p$ such that $n<p<n!$.

QED





\section*{1.3.36}

\begin{stonk}

To prove that,
for any primes $p\geq 5, q\geq 5$,
$3\mid p^2-q^2$.

\hr

\begin{stonk}

To prove that,
for any prime $p$,
$p^2$ can be written as $3m+1$
for some integer $m$.

\hr

Suppose prime number $p\geq 5$.

According to the Division Algorithm,
$p$ can be written as either $3n$, $3n+1$, or $3n+2$
for some integer $n$.

If $p=3n$, this forms a contradiction
since $3\mid p$ and (since $p\geq 5$) $p\neq 3$.

If $p=3n+1$,
then $p^2=3(3n^2)+3(2n)+1=3(3n^2+2n)+1$.
Thus, $p^2=3m+1$ for some integer $m$.

If $p=3n+2$,
then $p^2=3(3n^2)+3(2n)+4=3(3n^2+2n+1)+1$.
Thus, $p^2=3m+1$ for some integer $m$.

\end{stonk}

Suppose primes $p\geq 5, q\geq 5$.

There exist integers $a$ and $b$
for which $p^2=3a+1$ and $q^2=3b+1$.
\eel{
p^2&=3a+1\\
q^2&=3b+1\\
p^2-q^2&=3a-3b\\
p^2-q^2&=3(a-b)
}
Since $a-b$ is an integer, $3\mid p^2-q^2$.

\end{stonk}

\begin{stonk}

To prove that,
for any primes $p\geq 5, q\geq 5$,
$8\mid p^2-q^2$.

\hr

According to the Division Algorithm,
$p$ can be written either as
$4n$, $4n+1$, $4n+2$, or $4n+3$.
Since $p$ is prime and greater than $2$,
$p=4n$ and $p=4n+2=2(2n+1)$ both lead to contradictions.
Thus $p=4n+1$ or $p=4n+3$.

Similarly, for some integer $m$,
$q=4m+1$ or $q=4m+3$.

If $p=4n+1$ and $q=4m+1$, then
\eel{
p^2&=16n^2+8n+1\\
q^2&=16m^2+8m+1\\
p^2-q^2&=16(n^2-m^2)+8(n-m)\\
&=8(2(n^2-m^2))+8(n-m)\\
&=8(2(n^2-m^2)+n-m)\\
8&\mid p^2-q^2
}

If $p=4n+3$ and $q=4m+3$, then
\eel{
p^2&=16n^2+24n+9\\
q^2&=16m^2+24m+9\\
p^2-q^2&=16(n^2-m^2)+24(n-m)\\
&=8(2(n^2-m^2))+8(3(n-m))\\
&=8(2(n^2-m^2)+3(n-m))\\
8&\mid p^2-q^2
}

Without loss of generality, swap $p$ and $q$
if $p=4n+1$ and $q=4n+3$.
\eel{
p&=4n+3\\
q&=4n+1\\
p^2&=16n^2+24n+9\\
&=8(2n^2+3n+1)+1\\
q^2&=16m^2+8m+1\\
&=8(2m^2+m)+1\\
p^2-q^2&=8(2n^2+3n+1-2m^2-m)\\
8\mid p^2-q^2
}

In all 3 cases, $8\mid p^2-q^2$.

\end{stonk}

Suppose primes $p\geq 5, q\geq 5$.

By earlier conclusions, $3\mid p^2-q^2$ and $8\mid p^2-q^2$.

Since $24=[3, 8]$, by problem 1.2.32, $24\mid p^2-q^2$.

QED


\section*{2.1.19}

Suppose $[a]=[b]$ in $\ZZ_n$.

Due to the division algorithm,
$a$ can be written as $nq_a+r_a$, and
$b$ can be written as $nq_b+r_b$,
such that $0\leq r_a<n$ and $0\leq r_b<n$.

\begin{gather*}
b\in [b]\land [a]=[b]\\
b\in [a]\\
a\equiv b\\
n\mid a-b\\
n\mid nq_a+r_a-nq_b-r_b\\
nz=nq_a+r_a-nq_b-r_b\\
n(z-q_a+q_b)=r_a-r_b\\
n\mid r_a-r_b
\end{gather*}

Since $r_a<n$ and $0\leq r_b$, $r_a-r_b$ can be at most n-1.
Since $0\leq r_a$ and $r_b<n$, $r_a-r_b$ must be at least -(n-1).
Since $\abs{r_a-r_b}<n$ and $n\mid r_a-r_b$, $r_a-r_b$ must be 0,
and $r_a=r_b$.

Let an integer $r$ be equal to $r_a$.
\eel{
a&=nq_a+r\\
b&=nq_b+r
}

With application of Euclid's algorithm,
\eelc{
(a,n)=(r,n)\\
(b,n)=(r,n)\\
(a,n)=(r,n)=(b,n)\\
(a,n)=(b,n)
}

QED

\end{document}
