\documentclass[12pt]{article}

%\AtEndDocument{\cleardoublepage}
%\AtEndDocument{% based in \cleardoublepage
%  \clearpage
%  \ifodd\value{page}\else
%    \thispagestyle{empty}%
%    \hbox{}\newpage
%  \fi
%}

\usepackage[margin=1.0in]{geometry} 
\usepackage{amsmath,amsthm,amssymb,amsfonts}
\usepackage{fancyvrb}
\usepackage{graphics}

\usepackage{enumitem}

\theoremstyle{definition}
\newtheorem*{prob}{Problem}
\newtheorem*{soln}{Solution}

\DeclareMathOperator{\lcm}{lcm}

\newcommand{\CC}{{\mathbb{C}}}
\newcommand{\ZZ}{{\mathbb{Z}}}
\newcommand{\RR}{{\mathbb{R}}}
\newcommand{\NN}{{\mathbb{N}}}
\newcommand{\QQ}{{\mathbb{Q}}}
\newcommand{\PP}{{\mathbb{P}}}


\renewcommand{\mod}{\operatorname{mod}}

\begin{document}

% T I T L E
%
\title{421 HW 13 Group}
\author{\textbf{Put names here}}

\date{}

\maketitle

\textbf{NOTE}: Unless stated otherwise, $G$ is a (multiplicative) 
group with identity element $e$.



\begin{prob}[8.2.20] \text{ }

(a)   Let $N$ and $K$ be subgroups of a group $G$. 
If $N$ is normal in $G$, prove that $NK = \{nk \mid n\in N, k\in K\}$ 
is a subgroup of $G$. [Compare Exercise 26(b) of Section 7.3.]     

(b) If  both $N$ and $K$ are normal subgroups of  $G$, prove that $NK$ is normal.
\end{prob}

\begin{soln}\quad

(a)

normality: $\forall n\in N,g\in G\quad gng^{-1}\in N$

To prove: $e_G\in NK$

Since $N$ and $K$ are subgroups of $G$, $e_G\in N$, and $e_G\in K$.

Since $e_G\in N$ and $e_G\in K$, $e_Ge_G\in NK$.

$e_G\in NK$

To prove: $NK$ is closed under inverses

Let $nk\in NK$, where $n\in N$ and $k\in K$.

$N$ and $K$ are both groups, so $n^{-1}\in N$ and $k^{-1}\in K$.

Since $n^{-1}\in N$, $N$ is a normal subgroup of $G$, and $k^{-1}\in G$,
$k^{-1}n^{-1}k\in N$.

Since $k^{-1}n^{-1}k\in N$ and $k^{-1}\in K$,
$k^{-1}n^{-1}kk^{-1}\in NK$.

Equivalently, $k^{-1}n^{-1}=(nk)^{-1}\in NK$.

$NK$ is closed under inverses.

To prove: $NK$ is closed under multiplication

Let $nk,mj\in NK$, for some $n,m\in N$ and $k,j\in K$.

Since $N$ is normal, $m\in N$, and $k\in K$, $kmk^{-1}\in K$.

\begin{tabular}{c c c}
$\begin{gathered}
n\in N\quad kmk^{-1}\in N\\
\Downarrow\\
nkmk^{-1}\in N
\end{gathered}$ & &
$\begin{gathered}
k,j\in K\\
\Downarrow\\
kj\in K
\end{gathered}$\\
&
$\begin{gathered}
nkmk^{-1} kj\in NK\\
nkmj\in NK\\
(nk)(mj)\in NK
\end{gathered}$
\end{tabular}

$NK$ is closed under multiplication.

(b)

Let $nk\in NK$, for some $n\in N$ and $k\in K$.

Let $g$ be an arbitrary element of $G$.

Since $N$ and $K$ are normal, $gng^{-1}\in N$ and $gkg^{-1}\in K$.

By the definition of $NK$, $gng^{-1}gkg^{-1}\in NK$.

By inverses, $gnkg^{-1}\in NK$.

$NK$ is normal.

\end{soln}




\begin{prob}[8.3.9]
Let $G = \ZZ_6 \times \ZZ_2$ and let $N$ be the cyclic subgroup $\langle (1, 1) \rangle$. 
Describe the quotient group $G/N$. [That is, what well-known group $G/N$ isomorphic to?]
Justify.
\end{prob}

\begin{soln}

$|G/N|=[G:N]=\frac{|G|}{|N|}=\frac{12}{6}=2$.

All groups of order 2 are isomorphic to $\ZZ_2$.

$G/N$ is isomorphic to $\ZZ_2$.

\end{soln}

\begin{prob}[8.3.25] \text{ }

(a) Find the order of  $\frac{8}{9}$, $\frac{14}{5}$, and 
$\frac{48}{28}$ in the additive group $\QQ/\ZZ$.

(b) Prove that every element of  $\QQ/\ZZ$ has finite order.     

(c) Prove that $\QQ/\ZZ$ contains elements of  every possible finite order.
\end{prob}

\begin{soln}\quad

The identity element is $\ZZ$.

(a)

$|\frac89|=9$, $|\frac{14}5=5$, $|\frac{48}{28}|=|\frac{12}7|=7$.

(b)

Let $\frac ab$ be an arbitrary element of $\QQ/\ZZ$,
for some $a\in \ZZ$ and $b\in\ZZ^+$.

$b\frac ab=a\in\ZZ$, so $|\frac ab|\mid b$.

$|\frac ab|\mid b$, and $b>0$, so $|\frac ab|\leq b$

The order of this element is less than a finite integer, and so must be finite.

Since this element was arbitrary, the order of any element is finite.

(c)

Let $b\in\ZZ^+$.

Consider the element $\frac1b$.

$b\frac1b=1\in\ZZ$, so the order of $\frac1b$ must divide $b$.

For every positive integer $a$ such that $a<b$, $0<a\frac 1b<1$.

This means that $a\frac 1b$ is not an integer, so $\frac1b$
cannot be of order less than $b$.

The only integer which divides $b$ and is not less than $b$ is $b$.

$\frac1b$ must be of order $b$.

Since $b$ was an arbitrary element of $\ZZ^+$,
for every finite order, there exists an element with that order.

\end{soln}

\begin{prob}[8.4.18]
Find all homomorphic images of  $D_4$.
In other words, if $f:D_4 \to H$ is a surjective homomorphism,
then what are all the possibilities for $H$, up to isomorphism? [Hint: 
First Isomorphism Theorem.]
\end{prob}

\begin{soln}

\end{soln}

\begin{prob}[8.4.26]
Prove that $(\ZZ \times \ZZ)/\langle (2, 2) \rangle \cong \ZZ \times \ZZ_2$. 
[Hint: Show that $f:\ZZ \times \ZZ \to \ZZ \times \ZZ_2$, given by 
$f ((a, b)) = (a - b, [b]_2)$, 
is a surjective homomorphism.]
\end{prob}

\begin{soln}

\end{soln}


\begin{prob}[3.3.16]
Let $T$, $R$, and $F$ be the four-element rings whose tables are given in Example 5 of  
Section 3.1 and in Exercises 2 and 3 of  Section 3.1. 
Show that no two of  these rings are isomorphic.

For convenience, here are their operation tables:

$T = \{z,r,s,t \}$

\[
\begin{array}{c|cccc}
+ & z & r & s & t \\
\hline
z      & z & r & s & t  \\
r      & r & z & t & s  \\
s      & s & t & z & r  \\
t      & t & s & r & z 
\end{array}
\qquad\qquad
\begin{array}{c|cccc}
\cdot & z & r & s & t \\
\hline
z      & z & z & z & z  \\
r      & z & z & r & r  \\
s      & z & z & s & s  \\
t      & z & z & t & t 
\end{array}
\]

$R = \{ 0, e, b, c \}$

\[
\begin{array}{c|cccc}
+ & 0 & e & b & c \\
\hline
0      & 0 & e & b & c  \\
e      & e & 0 & c & b  \\
b      & b & c & 0 & e  \\
c      & c & b & e & 0 
\end{array}
\qquad\qquad
\begin{array}{c|cccc}
\cdot & 0 & e & b & c \\
\hline
0      & 0 & 0 & 0 & 0  \\
e      & 0 & e & b & c  \\
b      & 0 & b & b & 0  \\
c      & 0 & c & 0 & c 
\end{array}
\]

$F = \{ 0, e, a, b \}$

\[
\begin{array}{c|cccc}
+ & 0 & e & a & b \\
\hline
0      & 0 & e & a & b  \\
e      & e & 0 & b & a  \\
a      & a & b & 0 & e  \\
b      & b & a & e & 0 
\end{array}
\qquad\qquad
\begin{array}{c|cccc}
\cdot & 0 & e & a & b \\
\hline
0      & 0 & 0 & 0 & 0  \\
e      & 0 & e & a & b  \\
a      & 0 & a & b & e  \\
b      & 0 & b & e & a 
\end{array}
\]
\end{prob}

\begin{soln}

\end{soln}

\begin{prob}[3.3.34]
If  $f:R \to S$ is an isomomorphism of  rings, which of  the following properties are 
preserved by this isomorphism? Justify your answers.     

(a) $a\in R$ is a zero divisor.

(b) $a\in R$ is idempotent. (That is, $a^2=a$.)

(c) $R$ is an integral domain.
\end{prob}

\begin{soln}

\end{soln}

\begin{prob}[3.3.38]
Let $F$ be a field and $f:F \to R$ a homomorphism of  rings.     

(a)   If there is a nonzero element $c$ of  $F$ such that $f(c) = 0_R$, 
prove that $f$ is the zero homorphism (that is, $f(x) = 0_R$ for every $x\in F$). 
[Hint: $c^{-1}$ exists (Why?). If  $x\in F$, consider $f(xcc^{-1})$.]     

(b)   Prove that $f$ is either injective or the zero homomorphism. 
[Hint: If  $f$ is not the zero homomorphism and $f(a) = f(b)$, then $f(a - b) = 0_R$.]
\end{prob}

\begin{soln}

\end{soln}


\begin{prob}[3.3.42]
If  $(m, n) \neq 1$, prove that $\ZZ_{mn}$ is not isomorphic to $\ZZ_m \times \ZZ_n$.
\end{prob}

\begin{soln}

\end{soln}

\end{document}
