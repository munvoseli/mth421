\documentclass[12pt]{article}

%\AtEndDocument{\cleardoublepage}
%\AtEndDocument{% based in \cleardoublepage
%  \clearpage
%  \ifodd\value{page}\else
%    \thispagestyle{empty}%
%    \hbox{}\newpage
%  \fi
%}

\usepackage[margin=1.0in]{geometry} 
\usepackage{amsmath,amsthm,amssymb,amsfonts}
\usepackage{fancyvrb}
\usepackage{graphics}

\usepackage{enumitem}

\theoremstyle{definition}
\newtheorem*{prob}{Problem}
\newtheorem*{soln}{Solution}

\DeclareMathOperator{\lcm}{lcm}

\newcommand{\CC}{{\mathbb{C}}}
\newcommand{\ZZ}{{\mathbb{Z}}}
\newcommand{\RR}{{\mathbb{R}}}
\newcommand{\NN}{{\mathbb{N}}}
\newcommand{\QQ}{{\mathbb{Q}}}
\newcommand{\PP}{{\mathbb{P}}}


\renewcommand{\mod}{\operatorname{mod}}

\begin{document}

% T I T L E
%
\title{421 HW 9 Group}
\author{\textbf{change this to your names}}

\date{}

\maketitle

\textbf{NOTE}: Unless stated otherwise, $G$ is a (multiplicative) 
group with identity element $e$.

\textbf{Hint}: Some of these problems are related.

\begin{prob}[8.1.21]
Let $H$ and $K$, each of prime order $p$, be subgroups of a group $G$. 
If  $H \neq K$, prove that $H \cap K = \langle e \rangle$.
\end{prob}

\begin{soln}

\end{soln}

\begin{prob}[8.1.36]
Let $H$ and $K$ be subgroups of a finite group $G$ such that $[G:H] = p$ and $[G:K] = q$,
 with $p$ and $q$ distinct primes. Prove that $pq$ divides $[G:H \cap K]$.
\end{prob}

\begin{soln}

\end{soln}

\begin{prob}[8.1.37]
Let $G$ be an abelian group of order $n$ and let $k$ be a positive integer. 
If $(k, n) = 1$, prove that the function 
$f:G \to G$ given by $f(a)= a^k$ is an isomorphism. 
[Hint: To show $f$ is a bijection, find a formula for $f^{-1}$.]
\end{prob}

\begin{soln}

\end{soln}


\begin{prob}[8.1.40]
If  a prime $p$ divides the order of a finite group $G$, prove that the number of 
elements of order $p$ in $G$ is a multiple of  $p - 1$.
[Cauchy's Theorem says that the number of elements of order $p$ is positive.
However, you do not need to prove that or apply it. That is, for this exercise,
the proof will work fine if the number of elements of order $p$
is $0(p-1) = 0$.]
\end{prob}

\begin{soln}\quad

Outline:

Let $Y=\{a\in G\ \ \mid\ \ |a|=p\}$

Let $\sim$ be an equivalence relation on $Y$ such that
$a\sim b\Leftrightarrow b\in\langle a\rangle$
(conceptually, this should be equivalent to
$a\sim b\Leftrightarrow \langle a\rangle=\langle b\rangle$,
but this is not proven).

Each equivalence class is of size $p-1$ ($[a]=\langle a\rangle\setminus\{e\}$).

$Y$ is the union of these disjoint equivalence classes.

$|Y|$ must be a multiple of $p-1$.

This depends upon the theorem that any group of size $p$,
where $p$ is prime, is isomorphic to $\ZZ_p$.

Proof:

To prove $[a]\subseteq\langle a\rangle\setminus\{e\}$

$\begin{gathered}
b\in [a]\\
b\in \langle a\rangle\\
[a]\subseteq Y\\
|b|=p\\
b\neq e\\
b\in \langle a\rangle\setminus\{e\}
\end{gathered}$

To prove $\langle a\rangle\setminus\{e\}\subseteq[a]$

$\begin{gathered}
b\in\langle a\rangle\setminus\{e\}\\
|b|=p\\
b\in Y\\
b\in\langle a\rangle\\
b\in[a]
\end{gathered}$

Therefore, $\langle a\rangle\setminus\{e\}=[a]$.

To prove $\forall a\in Y\quad \langle a\rangle\cong\ZZ_p$

$\sim$ is reflexive: yeah

$\sim$ is symmetric:

$\begin{gathered}
b\in \langle a\rangle\\
\exists k\in\ZZ\quad b=a^k\\
b^{-k}=a\\
a\in \langle b\rangle
\end{gathered}$

$\sim$ is transitive:

$\begin{gathered}
a\sim b\\
b\sim c\\
\exists x,y\in \ZZ \quad b=a^x,c=a^y\\
c=a^{xy}\\
\end{gathered}$

$x$ cannot be a multiple of $p$, since, if it were, $b=e$.
$y$ can similarly not be a multiple of $p$.
By Euclid's lemma, $p\nmid xy$.
Thus, $a\sim c$.

Each equivalence class is of order $p-1$:

$e\not\in [a]$, so $[a][e]=\langle a\rangle$.

By closure in $G$, $\langle a\rangle\leq G$.

Since $|a|=p$, $|\langle a\rangle|=p$, so $\langle a\rangle\cong \ZZ_p$.

\end{soln}


\begin{prob}[8.1.41]
Prove that a group of  order $33$ contains an element of order $3$.
[Of course you are not allowed to apply Cauchy's Theorem.]
\end{prob}

\begin{soln}\quad

$\begin{gathered}
\forall g\in G\quad g^{11\cdot 3}=e\\
(g^{11})^3=e\\
|g^{11}|\mid 3\\
|g^{11}|=3\text{ or }|g^{11}|=1\\
|g^{11}|=3\text{ or }(g^{11}=e)\\
|g^{11}|=3\text{ or }(|g|=11\text{ or }g=e)\\
|g^{11}|=3\text{ or }|g|=11\text{ or }g=e\\
\end{gathered}$

It is unproven, but these are all mutually exclusive things.
For any given value of $g\in G$, exactly one of these things must be true.

By $8.1.41$, for some $n\in\ZZ$, the number of elements for which
$|g|=11$ is $10n$.

There is one element for which $g=e$.

Let the number of elements $g$ for which $|g^{11}|=3$ be denoted by $m$.

33 is the total number of elements, so

$\begin{gathered}
33=m+10n+1\\
33-10n-1=m\\
32-10n-1=m
\end{gathered}$

Since $m\geq 0$, and $n\geq 0$, the possibilities for $m$ are 32, 22, 12, and 2.

This means there exists $a\in G$ for which $|a^{11}|=3$.

$|a^{11}|$ is an element of order 3, so this is proven.


\end{soln}



\end{document}
