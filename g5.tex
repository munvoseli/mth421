\documentclass[12pt]{article}

%\AtEndDocument{\cleardoublepage}
%\AtEndDocument{% based in \cleardoublepage
%  \clearpage
%  \ifodd\value{page}\else
%    \thispagestyle{empty}%
%    \hbox{}\newpage
%  \fi
%}

\usepackage[margin=1.0in]{geometry} 
\usepackage{amsmath,amsthm,amssymb,amsfonts}
\usepackage{fancyvrb}
\usepackage{graphics}

\usepackage{enumitem}

\theoremstyle{definition}
\newtheorem*{prob}{Problem}
\newtheorem*{soln}{Solution}

\DeclareMathOperator{\lcm}{lcm}

\newcommand{\CC}{{\mathbb{C}}}
\newcommand{\ZZ}{{\mathbb{Z}}}
\newcommand{\RR}{{\mathbb{R}}}
\newcommand{\NN}{{\mathbb{N}}}
\newcommand{\QQ}{{\mathbb{Q}}}
\newcommand{\PP}{{\mathbb{P}}}


\renewcommand{\mod}{\operatorname{mod}}

\begin{document}

% T I T L E
%
\title{421 HW 5 Group}
\author{\textbf{change this to your names}}

\date{}

\maketitle

\textbf{Note}: $R$ denotes a ring and $F$ denotes a field and $p$ denotes a positive
prime number.

%\begin{prob}[4.1.5]
%Find polynomials $q(x)$ and $r(x)$ such that $f(x) = g(x)q(x) + r(x)$, 
%and $r(x) = 0$ or $\deg r(x) < \deg g(x)$:
%
%(d) $f(x) = 4x^4 + 2x^3 + 6x^2 + 4x + 5$ and $g(x) = 3x^2 + 2$ in $\ZZ_7[x]$.
%
%(The typography for the Division Algorithm is complex, 
%so just tell me the remainder at each step.
%For example, if $f(x) = x^3 + x + 5$ and $g(x) = x+1$ in $\QQ[x]$, the first
%remainder would be $-x^2+x+5$. The second remainder would be $2x+5$, and so on.
%Of course give the final $q(x)$ and $r(x)$.)
%\end{prob}
%
%\begin{soln}


%\end{soln}

%\begin{prob}[4.1.13]
%Let $R$ be a commutative ring. Let $f(x) = a_0 + a_1x + a_2x^2 + \dots + a_nx^n \in R[x]$.
% If  $a_n \neq 0_R$ and $f(x)$
%is a zero divisor in $R[x]$, prove that $a_n$ is a zero divisor in $R$.
%\end{prob}
%
%\begin{soln}
%
%\end{soln}


\begin{prob}[4.1.17]
Let $R$ be an integral domain. 
Assume that the Division Algorithm always holds in $R[x]$. Prove that $R$ is a field.
\end{prob}

\begin{soln}

Statement of the division algorithm:
given $f\in F[x]$ and $g\in F[x]$, there exist some polynomials $p$ and $r$
such that $f=gp+r$ and either $r=0$ or degree of $r$ is less than degree of $g$.

Suppose that $a$ is an arbitrary nonzero element of $R$.

By the Division Algorithm,
there exists some $p$ and $r$ for which $1=pa+r$,
where $r$ is either $0$ or has degree less than $a$.
$a$ has degree $0$, so it must be that $r=0$.
Thus, $1=pa$.
$a$ has an inverse.

Since $a$ was an arbitrary nonzero element of $R$,
every nonzero element of $R$ has a multiplicative inverse.
Therefore, $R$ is a field.

\end{soln}


%\begin{prob}[4.2.5 \& 6]
%(Read the fairly long intro to Exercise 5 in the book.)
%\#5 Use the Euclidean Algorithm to find the gcd of  the given polynomials.\\
%\#6 Express each of the gcd's in Exercise 5 as a linear combination of the two polynomials.
%
%(c) $x^4 + 3x^3 + 2x + 4$ and $x^2 - 1$ in $\ZZ_5[x]$.
%\end{prob}
%
%\begin{soln}
%
%\end{soln}

\begin{prob}[4.2.14]
Let $f(x), g(x), h(x)\in F[x]$, with $f(x)$ and $g(x)$ relatively prime. 
If  $f(x) \mid h(x)$ and $g(x) \mid h(x)$, prove that $f(x)g(x) \mid h(x)$.
\end{prob}

\begin{soln}

\end{soln}


\begin{prob}[4.3.12]
Express $x^4 - 4$ as a product of irreducibles in $\QQ[x]$, in $\RR[x]$, and in $\CC[x]$.
\end{prob}

\begin{soln}
\quad

\begin{tabular}{l l}
$\QQ[x]$ & $(x^2+2)(x^2-2)$\\
$\RR[x]$ & $(x^2+2)(x-\sqrt2)(x+\sqrt2)$\\
$\CC[x]$ & $(x-i\sqrt2)(x+i\sqrt2)(x-\sqrt2)(x+\sqrt2)$\\
\end{tabular}

\end{soln}


%\begin{prob}[4.3.15]
%(a) By counting products of  the form $(x + a)(x + b)$, 
%show that there are exactly $(p^2 + p)/2$ monic polynomials of 
%degree 2 that are \emph{not} irreducible in $\ZZ_p[x]$.  
%
%(b) Show that there are exactly $(p^2 - p)/2$ monic irreducible 
%polynomials of degree 2 in $\ZZ_p[x]$.
%\end{prob}
%
%\begin{soln}
%
%\end{soln}

%\begin{prob}[4.4.15]
%Prove that $x^2 + 1$ is reducible in $\ZZ_p[x]$ if and only if 
%there exist integers $a$ and $b$ such that $p = a + b$ and $ab \equiv 1 \pmod{p}$.
%\end{prob}
%
%\begin{soln}
%
%\end{soln}

\begin{prob}[4.4.16]
Let $f(x), g(x) \in F[x]$ have degree $\leq n$ and let 
$c_0, c_1, \dots, c_n$ be distinct elements of  $F$. If  $f(c_i) = g(c_i)$ for 
$i = 0, 1, \dots, n$, prove that $f(x) = g(x)$ in $F[x]$.
\end{prob}

\begin{soln}

For $i\in0,1,...,n$, it is said that $f(c_i)=g(c_i)$.
With subtraction, $f(c_i)-g(c_i)=0$.
Since the degree of $f$ and degree of $g$ are both $\leq n$,
it must be that $f-g=0$ or the degree of $f-g$ is $\leq n$.

If $f-g$ is nonzero,
since the degree of $f-g$ is $\leq n$,
then $f-g$ must have at most $n$ roots.
This is not the case, as it is said to have $n+1$ roots.

$f-g$ must therefore be the zero polynomial.

$f(x)-g(x)=0$, so $f(x)=g(x)$.

\end{soln}

\begin{prob}[4.4.19]
We say that $a \in F$ is a multiple root of  $f(x) \in F[x]$ if  
$(x - a)^k$ is a factor of  $f(x)$ for some $k \geq 2$. \\
(a)   Prove that $a \in \RR$ is a multiple root of  $f(x)\in \RR[x]$ 
if and only if  $a$ is a root of both $f(x)$ and $f'(x)$, where $f'(x)$ 
is the derivative of  $f(x)$.\\
(b)   If  $f(x) \in  \RR[x]$ and if $f(x)$ is relatively prime to $f'(x)$,
 prove that $f(x)$ has no multiple root in $\RR$.
\end{prob}

\begin{soln}

\textbf{(a)}

Suppose $x-a$ is a multiple root of polynomial $f(x)\in\RR[x]$.
This means that $(x-a)^k$ is a factor of $f$, for some $k\geq2$.
Let $f$ be rewritten as $g(x-a)^k$,
where $x-a$ does not divide $g$.

By the product rule of differentiation,
\begin{align*}
f&=g(x-a)^k\\
f'&=g'(x-a)^k + gk(x-a)^{k-1}\\
f'&=(x-a)^{k-1}(g'(x-a)^k+gk)\\
\end{align*}

If $k\geq 2$, then $k-1\geq 1$, so $x-a$ is a factor of $f'$.

This proves the forward direction.

For the backward direction,

Suppose that $f'=(x-a)g$ and $f=(x-a)h$ for some polynomials $g$ and $h$.

Through differentiation, $f'=(x-a)h'+h$.

\begin{align*}
f'&=(x-a)g\\
f'&=(x-a)h'+h\\
(x-a)g&=(x-a)h'+h\\
(x-a)(g-h')&=h
\end{align*}


Substituting into an earlier equation,
$f=(x-a)(x-a)(g-h')$.

$(x-a)^k$ is a factor of $f$ for some $k\geq 2$.
Therefore, $x-a$ is a multiple root.

The theorem is proven.

\textbf{(b)}



\end{soln}



\end{document}
