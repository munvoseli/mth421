\documentclass{article}
\nonstopmode
\usepackage[paperwidth=20cm,paperheight=575cm]{geometry}
\usepackage[framemethod=TikZ]{mdframed}
\usepackage{amsmath, amsfonts, amssymb}
\usepackage{mathtools} % abs

\geometry{margin=1cm}
\DeclarePairedDelimiter\abs{\lvert}{\rvert}
\setlength{\parindent}{0pt}
\setlength{\parskip}{\baselineskip}
\newcommand{\hr}{\vspace*{\parskip}\hrule}
\newcommand{\ZZ}{\mathbb{Z}}
\DeclareRobustCommand{\eel}[1]{\begin{align*}{#1}\end{align*}}
\DeclareRobustCommand{\eelc}[1]{\begin{gather*}{#1}\end{gather*}}
\newmdenv[roundcorner=2pt,align=center%
,linecolor=yellow!20%
,backgroundcolor=yellow!20%
,linewidth=4,outerlinewidth=2%
,outerlinecolor=blue!70!black%
]{stonk}


\AddToHook{shipout/before}{%
    \pdfpageheight=\pagetotal
    \advance\pdfpageheight by 2cm
}
\let\oldsection=\section
\def\section{\pagebreak\oldsection}
\AtEndDocument{\pagebreak}

\newcommand{\modn}{\text{ (mod $n$)}}
\newcommand{\Zn}{\ZZ_n}
\renewcommand{\land}{\text{ and }}
\renewcommand{\lor}{\text{ or }}

\begin{document}

\section*{2.3.9}


\eelc{
a\text{ is a unit in $\ZZ_n$}\\
\exists b\in\ZZ_n\quad ab\equiv 1\modn\\
n\mid ab-1\\
\Downarrow\\
\exists k\in\ZZ\quad ab-1=nk\\
ab-nk=1\\
ab+n(-k)=1\\
\gcd(a,n)=1
}


\eelc{
\forall c\in\Zn\setminus\{0\}\quad ac\neq 0\\
\Updownarrow\\
\nexists c\in\Zn\quad ac\equiv 0\\
\Uparrow\\
(c\in\Zn\land ac\equiv 0)\Rightarrow(c\equiv 0)
}

\eelc{
\gcd(a,n)=1\\
\exists k\in\ZZ,r\in\Zn\quad nk+r=c\\
\exists m\in\ZZ ac=mn\\
a(nk+r)=mn\\
ar=n(m-ak)\\
n\mid ar\\
\gcd(a,n)=1\\
n\mid r\text{ by Euclid's lemma}\\
r=0\\
c=nk\\
c\equiv 0\\
}

\end{document}
